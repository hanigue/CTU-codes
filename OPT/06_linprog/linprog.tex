\documentclass[10pt,a4paper,openright]{article}
\usepackage{amsfonts}
\usepackage{float}
\usepackage{hyperref}
\usepackage[utf8]{inputenc}
\usepackage[T1]{fontenc}
\usepackage{graphicx}
\usepackage{amsmath,amssymb,verbatim}
\textheight=240mm
\textwidth=170mm
\topmargin=-15mm
\oddsidemargin=-6mm
\evensidemargin=\oddsidemargin
\newcommand\tab[1][1cm]{\hspace*{#1}}
\usepackage{hyperref}
\newcommand{\norm}[1]{\left\lVert#1\right\rVert}

\begin{document}
	\begin{flushleft}
		\large Name: Matouš Dzivjak\\
		\large Date: 20.12.2018\\
	\end{flushleft}
\begin{center}
	\huge Lineární programování
\end{center}
\section{Zadání}
\begin{verbatim}
http://www.seas.ucla.edu/~vandenbe/ee236a/homework/problems.pdf#page=14&zoom=auto,-248,211
\end{verbatim}


\section{Řešení}

\subsection{Úkol a}

\textit{Definice úlohy LP, včetně přesné definice všech použitých proměnných (matic, vektorů). 
Pro zjednodušení zápisu uvažujte případ s N=2 stavy.}
\vspace{.4cm}

Ze zadání máme stavy $x(t) \in \mathbb{R}^n, t = 0,...,N$, a signály (akce)
$u(t) \in \mathbb{R}, t = 0,...,N-1$. Dynamický systém je pak definován jako:

\begin{equation}
    x(t+1)=Ax(t) + bu(t), t=0,...,N-1
\end{equation}

Kde $A \in \mathbb{R}^{n \times n}$ a $b \in \mathbb{R}^n$ jsou dány.
Počátečním stavem je stav $x(0) = 0$. 

Optimalizační úlohou minimílní spotřeby paliva (\textit{minimum fuel optimal control problem}) je zvolit 
signály $u(0),...,u(N-1)$ tak, abychom minimalizovali spotřebované palivo.
Což je dáno jako $F=\sum_{t=0}^{N-1}f(u(t))$ za podmínky $x(N)=x_{des}$.
Kde funkce $f: \mathbb{R} \to \mathbb{R}$ je mapa využití paliva pro signál (akci), 
tedy funkce, která vrací množství spotřebovaného paliva v závislosti na velikosti
signálu (akce). V této úloze je definována jako:

\begin{equation}\label{eq:fun}
f(a) = \begin{cases} 
    |a| & |a| \leq 1 \\
    2|a|-1 & |a| > 1 \\
 \end{cases}
\end{equation}

Definujeme úlohu lineárního programování (ze zadání pro $N=2$, nicméně je mi pohodlnější to napsat obecně,
a na $N=2$ se můžeme omezit kdykoliv). Optimalizujeme úlohu:

\begin{equation}
\min \sum_{t=0}^{N-1} f(u(t))
\end{equation}

Ze zadání víme, že platí:
\begin{equation}
	\begin{split}
		&x(1) = Ax(0) + bu(0) = A0 + bu(0) = bu(0)\\
		&x(2) = Ax(1) + bu(1) = Abu(0) + bu(1)\\
		&x(3) = Ax(2) + bu(2) = A^2bu(0) + Abu(1) + bu(2)\\
		&\vdots\\
		&x(N) = A^{N-1}bu(0) + A^{N-2}bu(1) + \hdots + Abu(N-1) + bu(N) = x_{des}
	\end{split}
\end{equation}

Kde $x(t)$ jsou hodnoty stavu v čase $t$ a $u(t)$ je signál v čase $t$. Z toho vytvoříme matici tak, abychom mohli optimální řešení zapsat jako násobek této matice
spolu s vektorem signálů v jednotlivých časech, tedy:

\begin{equation}\label{eq:defs}
	\begin{split}
	&C = \begin{bmatrix}A^{N-1}b & A^{N-2}b & \hdots & Ab & b \end{bmatrix}\\
	&u = \begin{bmatrix}u(0) & u(1) & \hdots & u(N-1)\end{bmatrix}^T
	\end{split}
\end{equation}

Dále musíme transformovat optimalizovanou funkci \ref{eq:fun}. 
To uděláme následovně. Přidáme novou proměnnou $d$ a podmínky:

\begin{equation}\label{eq:limits}
	\begin{split}
	&|a| \leq d\\
	&2|a| - 1 \leq d
	\end{split}
\end{equation}

To jsme mohli provést díky tomu, že na intervalu $[-1,1]$ je větší $|a|$,
zatím co na zbytku reálných čísel nabývá větší hodnoty $2|a| - 1$.
Proměnná $d$ nám tedy vždy omezuje obě rovnosti, ale první "se omezí" funkce,
která na daném intervalu odpovídá hodnotě předepsané funkcí $f$ a druhá bude nabývat hodnot nižší.

Nerovnosti \ref{eq:limits} zbavíme absolutní hodnoty.


\begin{equation}\label{eq:ineq}
	\begin{split}
	&-d \leq a \leq d\\
	&-\frac{d+1}{2} \leq a \leq \frac{d+1}{2}
	\end{split}
\end{equation}

Tyto nerovnosti musí platit pro každé $u(i)$ ($u(i)$ se stává při řešení optimalizační úlohy 
argumentem funkce \ref{eq:fun}, tedy ho dosazujeme do těchto nerovností). Zavedeme tedy vektor
$s$ (\textit{spotřeba}) $= \begin{bmatrix}s_0,\hdots,s_{N-1} \end{bmatrix}$,
s jedním $s_i$ pro každý signál (akci) $u(i)$ a ten použijeme v nerovnostech. Nyní už můžeme
definovat úlohu lineárního programování:

\begin{equation}
	\mathcal{LP} = 
	\begin{cases}
		\text{minimalizujeme} & 1^Ts\\
		\text{za podmínek} & -s \leq u \leq s\\
		& -\frac{s+1}{2} \leq u \leq \frac{s+1}{2}\\
		& Cu = x_{des}
	\end{cases}
\end{equation}

Kde:
\begin{equation}
	\begin{split}
		N \in \mathbb{N} & \hdots \text{časový horizont pro splnění úlohy}\\
		s \in \mathbb{R}^{N-1} & \hdots \text{spotřeba}\\
		u \in \mathbb{R}^{N-1} & \hdots \text{signály v jednotlivých časových krocích}\\
		C \in \mathbb{R}^{(N-1) \times n} & \hdots \text{matice \ref{eq:defs}} \\
		x_{des} \in \mathbb{R}^{n} & \hdots \text{cílový stav}
	\end{split}
\end{equation}

Tedy přeloženo do slov: Minimalizujeme sumu spotřeby ($1^Ts$), 
která nám omezuje signály (akce, které provádíme) v jednotlivých časových krocích (\ref{eq:ineq}) 
tak, abychom v čase $N$, tedy v zadaném časovém horizontu byly v cílovém stavu $x_{des}$.


\subsection{Úkol b}

\textit{Graf obsahující závislost $u(t)$, $x_1(t)$ a $x_2(t)$ na $t$. Optimální hodnotu spotřeby F.}
\vspace{.4cm}


\end{document}